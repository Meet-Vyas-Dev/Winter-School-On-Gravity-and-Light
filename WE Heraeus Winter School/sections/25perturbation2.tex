\chapter{Perturbation Theory II}

It is a fact that any small, but otherwise arbitrary, transformation of coordinates is given by choice of vector field $\xi$, whose component functions w.r.t. the original chart, $\xi^m_{(x)}$, are small. Then the change incurred by the metric components under such a small, but otherwise arbitrary, transformation is given by\footnote{The $\Delta_{\xi}$ here is not the Laplacian, but simply denotes the change under $\xi$.} 
\bse 
    \Delta_{\xi}g_{ab} = \big(\cL_{\xi}g\big)_{ab},
\ese 
where $\cL$ is the Lie derivative. In other words, if the $\del g_{ab}$ takes this form it is a `fake' perturbation, as described at the end of last lecture. But of course we write 
\bse 
    g_{ab} + \del g_{ab},
\ese 
and so we must study how $\del g_{ab}$ itself changes if we choose a small, arbitrary $\xi$. If it is a fake perturbation then we know the change in the metric components is of the above form, but $\del g_{ab}$ is the change in the metric, and so taking a further change $\xi$ will simply give us 
\bse 
    \big(\Delta_{\xi} \del g\big)_{ab} = \big(\cL_{\xi}g\big)_{ab},
\ese 
where we note that there is no $\del$ on the right-hand side. The aim is to find which components are fake and then work out which, if any, combinations of components cancel on the right-hand side giving us a real perturbation. 

\section{Calculate $\Delta_{\xi}a$, $\Delta_{\xi}c$,  $\Delta_{\xi}b_{\a}$ \& $\Delta_{\xi}e_{\a\beta}$}

Even though we did the decomposition in the last lecture to the capital letters ($A,B_{\a}$, etc.), it is actually cleverer to work out the change of the lower case letters first.

Our chart is $(U,x)$ where $g = dx^0\otimes dx^0 - \gamma_{\a\beta}dx^{\a}\otimes dx^{\beta}$. We can decompose the vector field $\xi$ in this chart as
\bse 
    \xi = T\p_0 +L^{\beta}\p_{\beta},
\ese 
where $T$ and $L^{\mu}$ are scalar fields that can depend on all the chart coordinates, including $x^0$. We then calculate the change of $\del g_{ab}$ incurred by $\xi$ by considering $T$ and $L^{\mu}$. We have 
\bse 
    \begin{split}
        \big(\Delta_{\xi}\del g\big)_{ab} & = \big(\cL_{\xi}g\big)_{ab} \\
        & = \xi^mg_{ab,m} + {\xi^m}_{,a}g_{mb} + {\xi^m}_{,b}g_{am} \\
        & = Tg_{ab,0} + L^{\mu}g_{ab,\mu} + T_{,a}g_{0b} + {L^{\mu}}_{,a}g_{\mu b} + T_{,b}g_{a0} + {L^{\mu}}_{,b}g_{a\mu}.
    \end{split}
\ese 
We now consider each of the components separately. Using a dot to indicate a derivative w.r.t. $x^0$, we have:
\bse
    \big(\Delta_{\xi}\del g\big)_{00} = 0 + 0 + \dot{T} + 0 + \dot{T} + 0 = 2\dot{T},
\ese 
where we have used the fact that $g_{00}=1$ and $g_{0\a}=0$. 

Next we have 
\bse 
    \big(\Delta_{\xi} \del\big)_{0\beta} = 0 + 0 + 0 - \dot{L}^{\mu} \gamma_{\mu\beta} + T_{,\beta} + 0 = -\dot{L}^{\mu}\gamma_{\mu\beta} + T_{,\beta},
\ese 
where we have used $g_{0\beta}=0$, $g_{00}=1$ and $g_{\a\beta} = -\gamma_{\a\beta}$. We can then use the fact that $\gamma_{\a\beta}$ is independent of $x^0$ to give us 
\bse 
    \big(\Delta_{\xi} \del\big)_{0\beta} = T_{,\beta} - \big(L^{\mu}\gamma_{\mu\beta}\big)_{,0} = D_{\beta}T - \dot{L}_{\beta},
\ese
where $D_{\beta}$ is the Levi-Civita covariant derivative using $\gamma$.

Finally we have 
\bse 
    \big(\Delta_{\xi}\del g\big)_{\a\beta} = 0 - L^{\mu} \gamma_{\a\beta,\mu} + 0 - {L^{\mu}}_{,\a}\gamma_{\mu\beta} + 0 - {L^{\mu}}_{,\beta} \gamma_{\a\mu}. 
\ese
We then use 
\bse 
    {L^{\mu}}_{,\a}\gamma_{\mu\beta} = \big(L^{\mu}\gamma_{\mu\beta}\big)_{\a} - L^{\mu} \gamma_{\mu\beta,\a}
\ese 
to give 
\bse 
    \big(\Delta_{\xi}\del g\big)_{\a\beta} = -2\bigg[ L_{(\a,\beta)} -\frac{1}{2}L^{\mu}\big( \gamma_{\mu\beta,\a} + \gamma_{\mu\a,\beta} - \gamma_{\a\beta,\mu}\big) \bigg] = -2 D_{(\a}L_{\beta)},
\ese 
where again $D_{\a}$ is the Levi-Civita covariant derivative using $\gamma$.

Using the expression for $\del g$ in terms of $a,b_{\a},c$ and $e_{\a\beta}$ from last lecture, we conclude that 
\bse 
    \begin{split}
        \Delta_{\xi}(2a) & = 2\dot{T}  \\
        \Delta_{\xi}(-b_{\a}) & = D_{\a}T - \dot{L}_{\a}\\
        \Delta_{\xi} \big(-2c\gamma_{\a\beta} -e_{\a\beta}\big) & = -2D_{(\a}L_{\beta)}.
    \end{split}
\ese 
The first two expressions clearly tell us that 
\bse 
    \Delta_{\xi} a = \dot{T}, \qand \Delta_{\xi} b_{\a} = \dot{L}_{\a} - D_{\a}T,
\ese 
but the third expression needs a little work. We contract both sides with $\gamma^{\a\beta}$ and use $\gamma^{\a\beta}\gamma_{\a\beta}=3$ along with the fact that $D$ is $\gamma$-metric compatible (so we can `take $\gamma$ inside it'), giving
\bse 
    6\Delta_{\xi}c + \Delta_{\xi}\gamma^{\a\beta}e_{\a\beta} = 2D_{\a}\big(\gamma^{\a\beta}L_{\beta}\big) \qquad \implies \qquad \Delta_{\xi} c = \frac{1}{3}D_{\a}L^{\a},
\ese 
where we used the trace-free condition of $e_{\a\beta}$. From this it follows that 
\bse 
    \Delta_{\xi} e_{\a\beta} = 2D_{(\a}L_{\beta)} - \frac{2}{3}D_{\mu}L^{\mu}\gamma_{\a\beta}.
\ese 

\section{Scalar Perturbations}

We now wan to consider the decomposed fields, and we start with the scalar perturbations.

\bcl 
    If we consider the seemingly restricted case $L^{\a}=D^{\a}L$ for some scalar field $L$, we actually get the same result for the scalar perturbations as if we had done it generally.
\ecl 

We do not prove this claim here but just use it to simplify the following. 

Using the above claim and the results from last lecture we see that 
\bse 
    \begin{split}
        \Delta_{\xi} A & = \dot{T}, \\
        \Delta_{\xi}B & = \dot{L}-T \\
        \Delta_{\xi} C & = \frac{1}{3}\Delta L \\
        \Delta_{\xi} E & = L,
    \end{split}
\ese 
where on the right-hand side of the $C$ equation $\Delta$ is the Laplacian.
\bbox
    Show the above results hold. 
    
    \textit{Hint: If you're stuck Dr. Schuller does this in the videos.}\footnote{There is a couple factors of $2$ for the  $E$ in my notes that I think Dr. Schuller missed, but they end up cancelling above so I get the same result for $\Delta_{\xi}E$.}
\ebox 

We can now use these results to show the results quoted at the end of last lecture, namely
\bse 
    \Psi := A + \dot{B} - \Ddot{E}, \qand \Phi := C - \frac{1}{3}\Delta E
\ese 
are gauge invariants. 

\bbox
    Show that above hold. 
    
    \textit{Hint: Again this is done in the lectures.}
\ebox 

\br 
    Note these results tell us that the scalars $A,B,C$ and $E$ can not be `switched on' independently, as the only real perturbations are given as combinations of them. 
\er 

For convenience only, and precisely because the $\Phi$ and $\Psi$ are gauge invariants, we are free to pick any $(T,L^{\a})$ we want. Such a choice is known as a \textbf{gauge}, and the process  \textbf{gauge fixing}. Again should be familiar from electromagnetism. We decide to use the gauge 
\bse 
    T := B - \dot{E}, \qand L := -E.
\ese 
We choose this gauge as then 
\bse 
    \begin{split}
        \Delta_{\xi} E = L \qquad \implies \qquad  E & = 0 \\
        \Delta_{\xi} B = \dot{L} - T \qquad \implies \qquad B & = 0,
    \end{split}
\ese 
and so 
\bse 
    \Psi = A, \qand \Phi = C.
\ese 
This is known as the \textbf{longitudinal gauge}. Working in this gauge, we finally get the left-hand side for the scalar perturbations for Einsteins equations, $\del G_{ab} = T_{ab}$, as 
\bse 
    \begin{split}
        \del G_{00} & = 2\Delta \Phi \\
        \del G_{0\a} & = 2 D_{\a} \dot{\Phi} \\
        \del G_{\a\beta} & = \bigg[2\Ddot{\Phi} - \frac{2}{3}\Delta(\Psi+\Phi)\bigg]\gamma_{\a\beta} + \bigg[ D_{(\a}D_{\beta)} - \frac{1}{3}\gamma_{\a\beta} \Delta\bigg](\Psi+\Phi)
    \end{split}
\ese 

\section{Vector Perturbations}

\bcl 
    As with the scalar perturbations, it turns out we can consider the seemingly restricted cases of 
    \bse 
        T = 0, \qand D_{\a}L^{\a}=0, 
    \ese 
    and get the full result.
\ecl 
Again, we do not prove this claim but just use it. 

The results here are 
\bse 
    \Delta_{\xi} B_{\a} = \dot{L}_{\a}, \qand \Delta_{\xi}E_{\a} = L_{\a},
\ese 
and so the only gauge invariant quantities are
\bse 
    \Theta := B_{\a} - \dot{E}.
\ese    

We then use the \textbf{vector gauge}, which sets $E=0$, and obtain $\Theta_{\a} = B_{\a}$. We then obtain the left-hand side of the vector perturbations of Einsteins equations, 
\bse 
    \del G_{0\a} = \frac{1}{2}\Delta \Theta_{\a}, \qand \del G_{\a\beta} = D_{(\a} \dot{\Theta}_{\beta)}.
\ese 

\section{Tensor Perturbations}

\bcl 
    Once again we can consider a seemingly restricted case, this time $T=0 = L^{\a}$. 
\ecl 
With this claim it follows trivially that 
\bse 
    \Delta_{xi} E_{\a\beta} =0,
\ese 
and so all $E_{\a\beta}$ are gauge invariants. The left-hand side equations are 
\bse 
    \del G_{\a\beta} = - \ddot{E}_{\a\beta} + \Delta E_{\a\beta}.
\ese

\br 
    If we start from the exact solution $G_{ab}[g]=0$ and do not introduce any mass (so $\del T_{ab}[g]=0$) we get for the tensor perturbations 
    \bse 
        -\ddot{E}_{\a\beta} + \Delta E_{\a\beta} = 0,
    \ese 
    which is a \textit{wave equation}. These are so-called \textbf{gravitational waves}. It is important to note that this wave equation is made from gauge invariant quantities and is a real world wave, not some `coordinate wave' that can be removed by a transformation.
\er 